\documentclass[11pt,a4paper]{article}
\usepackage{isabelle,isabellesym}
\usepackage[utf8]{inputenc}  % Input encoding
\usepackage[american]{babel} % Language
\usepackage[defblank]{paralist} % for compact lists
\usepackage{amsmath}
\usepackage{amssymb}
\usepackage{amsthm}
\usepackage{stmaryrd}
\usepackage{verbatim}
\usepackage{dot2texi}
\usepackage{pdfpages}
\newtheorem{definition}{Definition}[section]
\newtheorem{theorem}{Theorem}[section]
\newtheorem{lemma}{Lemma}[section]
\newcommand{\definitionautorefname}{Definition}

% this should be the last package used
\usepackage{pdfsetup}
% urls in roman style, theory text in math-similar italics
\urlstyle{rm}
\isabellestyle{it}

% for uniform font size
% \renewcommand{\isastyle}{\isastyleminor} % Everything bigger
%\renewcommand{\isastyle}{\isastyleminor}

%========= DRAFT ONLY ===============
\makeatletter
\newcommand\CO[1]{%
  \@tempdima=\linewidth%
  \advance\@tempdima by -2\fboxsep%
  \advance\@tempdima by -2\fboxrule%
  \leavevmode\par\noindent%
  \fbox{\parbox{\the\@tempdima}{\small\sf #1}}%
  \smallskip\par}
\newcommand\NOTE[2][Note]{%
	\leavevmode\marginpar{\raggedright\hangindent=1ex\small\textbf{#1: }#2}}
\newcommand\OLD[1]{%
    \slshape[\textbf{old: }\ignorespaces #1\unskip]}

%======= END DRAFT ONLY =============

\title{A Formalization of
Assumptions and Guarantees for Compositional Noninterference}
\author{Sylvia Grewe, Heiko Mantel, Daniel Schoepe}
\begin{document}
\maketitle
% sane default for proof documents
\parindent 0pt\parskip 0.5ex

\begin{abstract}
  Research in information-flow security aims at developing methods to
  identify undesired information leaks within programs from private
  (high) sources to public (low) sinks. For a concurrent system, it is
  desirable to have compositional analysis methods that allow for
  analyzing each thread independently and that nevertheless guarantee
  that the parallel composition of successfully analyzed threads
  satisfies a global security guarantee. However, such a compositional
  analysis should not be overly pessimistic about what an environment
  might do with shared resources. Otherwise, the analysis will reject
  many intuitively secure programs.

  The paper "Assumptions and Guarantees for Compositional
  Noninterference" by Mantel et. al. \cite{conf/csfw/MantelSS11}
  presents one solution for this problem: an approach for
  compositionally reasoning about non-interference in concurrent
  programs via rely-guarantee-style reasoning.  We present an
  Isabelle/HOL formalization of the concepts and proofs of this
  approach.

The formalization includes the following parts:

\begin{compactitem}
\item Notion of SIFUM-security and preliminary concepts:\\
  \texttt{Preliminaries.thy}, \texttt{Security.thy}
\item Compositionality proof: \texttt{Compositionality.thy}
\item Example language: \texttt{Language.thy}
\item Type system for ensuring SIFUM-security and soundness proof: \\
  \texttt{TypeSystem.thy}
\item Type system for ensuring sound use of modes and soundness proof:
  \texttt{LocallySoundUseOfModes.thy}
\end{compactitem}

\end{abstract}

\tableofcontents

\input{Preliminaries.tex}

\input{Security.tex}

\input{Compositionality.tex}

\input{Language.tex}

\input{TypeSystem.tex}

\input{LocallySoundModeUse.tex}

\bibliography{root}
\bibliographystyle{alpha}
\end{document}

%%% Local Variables:
%%% mode: latex
%%% TeX-master: t
%%% End:
