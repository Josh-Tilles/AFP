\chapter{Introduction}
  This development provides an efficient, extensible, machine checked collections framework for use
  in Isabelle/HOL. The library adopts the concepts of interface, implementation and generic algorithm
  known from object oriented (collection) libraries like the C++ Standard Template Library\cite{C++STL} or 
  the Java Collections Framework\cite{JavaCollFr} and makes them available in the Isabelle/HOL environment.

  The library uses data refinement techniques to refine an abstract specification (in terms of high-level concepts such as sets) to a more concrete implementation (based on collection datastructures like red-black-trees).
  This allows algorithms to be proven on the abstract level at which proofs are simpler because they are not cluttered with low-level details.

  The code-generator of Isabelle/HOL can be used to generate efficient code in all supported target languages, i.e. Haskell, SML, and OCaml.

  For more documentation and introductory material refer to the userguide (Section~\ref{thy:Userguide}) and the ITP-2010 paper \cite{LammichLochbihler2010ITP}.

\section{Document Structure}
  Chapter~\ref{ch:spec} contains the abstract specifiation of the collections, which are (ordered) maps, (ordered) sets, sequences, finger trees and (unique) priority queues.
  In chapter~\ref{ch:GA}, generic algorithms implement operations on these collections and adapters between them.
  Chapter~\ref{ch:impl} contains the concrete implementations of the data structures and their integration with the Isabelle Collectons Framework.
  There are implemenations for maps and sets using (associative) lists, red-black trees, hashing and tries.
  Implementations for finger trees and priority queues can be found an AFP entry of its own. Chapter~\ref{ch:impl} also contains an overview of all interfaces and
  implementations.
  Chapter~\ref{thy:Userguide} provides a userguide to the Isabelle Collections Framework.
  Some examples can be found in chapter~\ref{ch:examples}.
  Finally, a short conclusion and outlook to further work is given in \ref{ch:conclusion}.

